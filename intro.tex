\section*{Introduction}


Thanks for using Overleaf to write your article. Your introduction goes here! Some examples of commonly used commands and features are listed below, to help you get started. Leave a blank line between blocks of text to start a new paragraph---use \verb|\\| for separating tabular rows or hard line-breaks only. Abbreviations should be defined in the text at first mention.

Please also take note of the \verb|\section*{...}| titles in this template: they are the required sections in a regular research Article manuscript. 

In particular, the main text of regular Articles and Computational Tools manuscripts must be structured with the following sections: \textbf{Introduction}, \textbf{Materials and Methods}, \textbf{Results}, \textbf{Discussion (or Results and Discussion)}, \textbf{Conclusion}.

Theoretical manuscripts may include just a \textbf{Methods} section and do not require \textbf{Materials}.

No particular organization structure is required for Letters.

If your manuscript is accepted, the Biophysical production team will re-format the references for publication. \emph{It is not necessary to format the reference list yourself to mirror the final published form.}

\section*{Materials and Methods}

Capitalize trade names and give manufacturers' full names and addresses (city and state). 


\subsection*{Sectioning commands}

Use \verb|\section*{...}| and \verb|\subsection*{...}| to create first- and second-level headings. Sed ut perspiciatis unde omnis iste natus error sit voluptatem accusantium doloremque laudantium, totam rem aperiam, eaque ipsa quae ab illo inventore veritatis et quasi architecto beatae vitae dicta sunt explicabo. 

\subsection*{Figures and Tables}

Use the table and tabular commands for basic tables --- see Table \ref{tab:widgets}, for example. \href{http://tablesgenerator.com}{TablesGenerator.com} is a handy tool for designing tables and generating the \LaTeX{} \texttt{tabular} code, which you can copy and paste into your article here.

You can upload a figure (JPG, PNG or PDF) using the PROJECT menu (Files\ldots > Add files). To include it in your document, use the \verb|graphicx| package and the \verb|\includegraphics| command as in the code for Figure \ref{fig:view}. 

In addition, you can use \verb|\ref{...}| and \verb|\label{...}| commands for cross-references.

\begin{table}[hbt!]
\caption{An example table}
\label{tab:widgets}
\centering

\begin{threeparttable}

\begin{tabular}{c l r}
\hline
Code & Item & Quantity \\\hline
W1 & Widgets\tnote{a} & 42 \\
G35 & Gadgets & 13\tnote{b} \\
\hline
\end{tabular}

\begin{tablenotes}
\item[a] This is a table note.
\item[b] This is another table note.
\end{tablenotes}

\end{threeparttable}

\end{table}

\begin{figure}[hbt!]
\centering
\includegraphics[width=0.6\linewidth]{example-image}
\caption{A figure example.}
\label{fig:view}

\end{figure}